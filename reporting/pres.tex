\documentclass[]{beamer}

\usepackage[utf8]{inputenc}
\usepackage[T1]{fontenc}
\usepackage{natbib}
\usepackage[english]{babel}
\usepackage{graphicx}
\usepackage{xcolor} %Couleurs
\usepackage{tikz}
%\usepackage[left=0.2cm]{geometry}
% \usepackage[colorlinks=true,linkcolor=black,citecolor=black,urlcolor=black]{hyperref}
\usetikzlibrary{arrows}
\usepackage{fancybox}
\usepackage{setspace}
\usepackage{datetime}

\DeclareUnicodeCharacter{00A0}{~} % Bug caractère \u8

\usecolortheme{bop}
\useoutertheme{bop}
\useinnertheme{bop}

% \AtBeginSection[]
% {
%   \addtocounter{framenumber}{-1}
%   \begin{frame}
%     \frametitle{Plan}
%     \tableofcontents[currentsection]
%   \end{frame}
% }

\newlength{\bopwidth}

\newdateformat{specialdate}{\twodigit{\THEDAY}/\twodigit{\THEMONTH}/\THEYEAR}

\author{Xavier Laviron}
\title[Lab Meeting]{
  Evaluation of a new method to describe weed communities
}
\date{\today}
\institution{Master BOP, Dijon}

\newcommand\bopbox[1]{%
  \settowidth{\bopwidth}{\Large\bfseries#1}
  \colorbox{bop}{\parbox[c][][c]{\bopwidth}{\textcolor{white}{\Large\bfseries#1}}}
}

\begin{document}

\frame[plain]{\titlepage}

%-------------------------------------------------------------------------------

% \begin{frame}
%   \frametitle{Plan}
%   \tableofcontents
% \end{frame}

%-------------------------------------------------------------------------------

\section{Contexte}

\begin{frame}
  \frametitle{Context}

  \begin{block}{General context}
    Estimate the impact of weeds on crop yield
  \end{block}

  \begin{block}{Problems}
    \begin{itemize}
      \item > 200 weed species in our datasets
      \item Lots of variables for each plot
    \end{itemize}
  \end{block}

  \begin{block}{Solution: reduce the number of variable}
    \begin{itemize}
      \item Consider only a few species: which ones?
      \item Use biodiversity indices (richness, Simpson's): we loose
        informations about species
      \item \textbf<1->{\textcolor<1->{greenbop}{Create groups of species to
        describe the community}}
    \end{itemize}
  \end{block}

\end{frame}

% ------------------------------------------------------------------------------

\begin{frame}
  \frametitle{How to create groups?}

  \begin{block}{Biological groups}
    \begin{itemize}
      \item Functional traits: How to choose adequate traits ?
      \item Phylogeny: can summarise informations given by all traits
    \end{itemize}
  \end{block}

  \begin{block}{Statistical groups}
    \begin{itemize}
      \item 'Classic' clustering
      \item 'Soft' clustering
    \end{itemize}
  \end{block}

\end{frame}

% ------------------------------------------------------------------------------

\begin{frame}
  \frametitle{Categorisation of nature is a tough job}

  \centering\includegraphics<1>[scale=0.3]{img/groups1.pdf}
  \centering\includegraphics<2>[scale=0.3]{img/groups2.pdf}
  \centering\includegraphics<3>[scale=0.3]{img/groups3.pdf}
  \centering\includegraphics<4>[scale=0.3]{img/groups4.pdf}

\end{frame}

% ------------------------------------------------------------------------------

\section{Method}

\begin{frame}
  \frametitle{What do we propose?}

  \begin{block}{Latent Dirichlet Allocation (LDA)}
    \begin{itemize}
      \item Initially described by \citet{blei2003latent} for text mining
        analyses
      \item \citet{valle2014decomposing} propose an analogy with biodiversity
        data
      \item Create groups based on relative abundance of each species
      \item Can describe each community as a combination of component
        communities
    \end{itemize}
  \end{block}

\end{frame}

% ------------------------------------------------------------------------------

\begin{frame}
  \frametitle{La méthode LDA}

  \centering\includegraphics<1>[scale=0.4]{img/lda1.pdf}
  \centering\includegraphics<2>[scale=0.4]{img/lda2.pdf}
  \centering\includegraphics<3>[scale=0.4]{img/lda3.pdf}

\end{frame}

% ------------------------------------------------------------------------------

\section{Conclusion}

\begin{frame}
  \frametitle{Summary}

  \begin{block}{Advantages}
    \begin{itemize}
      \item Describe continuous changes between communities
      \item More parcimonious than other clustering methods (less groups might
        be needed to fit the data, according to \citet{valle2014decomposing})
    \end{itemize}
  \end{block}

  \begin{block}{Drawbacks}
    \begin{itemize}
      \item Need abundance data
      \item The number of groups have to be decided \textit{a priori}
    \end{itemize}
  \end{block}

\end{frame}

% ------------------------------------------------------------------------------

\begin{frame}
  \frametitle{Next steps}

  \begin{itemize}
    \item Run the model on data from Chiz\'e
    \item Determine the adequate number of groups
    \item Check if the groups are meaningfull
  \end{itemize}

\end{frame}

% -----------------------------------------------------------------------------

\begin{frame}
  \frametitle{References}

  \begin{small}
    \bibliographystyle{apalike}
    \bibliography{biblio}
  \end{small}

\end{frame}

\end{document}
