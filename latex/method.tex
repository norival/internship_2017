\documentclass[a4paper, 12pt]{article}

\usepackage[utf8]{inputenc}
\usepackage[T1]{fontenc}
\usepackage[francais]{babel}
\usepackage[top=2.5cm, bottom=2.5cm, left=2.5cm, right=2.5cm]{geometry}
\usepackage{graphicx}
\usepackage{xcolor} %Couleurs
\usepackage{icomma}
\usepackage[colorlinks=true,linkcolor=black,citecolor=black,urlcolor=black]{hyperref}
\usepackage{tikz}
\usepackage{tabularx}
\usepackage{parskip}
\usepackage{amsfonts}

\begin{document}

\section{Estimation des abondances}

On suppose que la distribution des espèces dans la parcelle suit une loi de
Poisson de paramètre $\lambda$.
On a donc :

\[
  P(X = k) = \frac{\lambda^k}{k!} e^{-\lambda}
\]

Avec :
\begin{itemize}
  \item $P(X = k)$: la probabilité d'observer les données $k$, ou la
    vraissemblance des données observées
  \item $\lambda$ le paramètre de la loi de Poisson, qui correpsond au nombre
    d'événements apparaissant dans l'intervalle considéré. Dans notre cas, c'est
    le nombre de fois où la plante apparaît dans un quadra (ou sous-quadra)
\end{itemize}

Sachant $k$ (le vecteur d'abondance de l'espèce $i$ dans la parcelle $j$), on
cherche $\lambda$ tel que $P(X=k)$ soit maximum.
Pour cela, on cherche à maximiser la fonction de log-vraissemblance $\mathrm{U}
= \ln(P)$ avec $ P = f(\lambda)$, définie sur $\mathbb{N}^*$ par :

\[
  \mathrm{U} = \ln f(\lambda) = k\ln \lambda - \ln k! - \lambda
\]

Maximiser la fonction U revient à minimiser la fonction $-\mathrm{U}$.
On cherche donc $\lambda$ tel que la fonction $-\mathrm{U}$ soit minimale.

Une fois $\lambda$ obtenu, on calcule $-1 \times\lambda / s$, $s$ étant la
surface d'un individu statistique (quadra ou sous-quadra), afin d'obtenir la
densité de plantes en $\mathrm{m}^{-2}$.

\subsection{Code R}

\begin{small}
\begin{verbatim}
# On minimise la focntion -U entre -50 et +50, 'h.fct' est la fonction qui doit
# être minimisée et v1 le vecteur d'abondance considéré.
Zu <- nlminb(c(0, 0), h.fct, v = v1, lower = c(-50, -50), upper = c(50, 50))

# Zu contient les 2 bornes encadrant le paramètre lambda pour lequel la fonction
# est minimale.
# On calcule la moyenne géométrique de ces bornes afin d'obtenir le paramètre
# optimal. On passe en exponentielle car la fonction minimisée correspond à la
# log-vraissemblance
mm <- com.mean(exp(Zu$par[1]), exp(Zu$par[2]))
\end{verbatim}

\begin{verbatim}
h.fct <- function(ltheta, v = v) {
  # v = observations
  # ltheta = log du paramètre de poisson

  library(compoisson)
  theta <- exp(ltheta)

  # si le paramètre est énorme, pas possible
  if(max(theta)>30){return(100000)}

  # proba d'abondance pour les espèces ayant un indice d'abondance de 0
  lp0 <- com.log.density(0, theta[1], theta[2])

  # proba d'abondance pour les espèces ayant un indice d'abondance de 1
  lp1 <- com.log.density(1, theta[1], theta[2])

  # proba d'abondance pour les espèces ayant un indice d'abondance de 2: 1 - la
  # somme des 2 autres
  lp2 <- log(1 - exp(lp0) - exp(lp1))
  lp <- c(lp0, lp1, lp2)
  ll <- (-1)*sum(lp[v+1])

  return(ll)
}
\end{verbatim}
\end{small}

\section{Classification des communautés adventices}

On définit :

\begin{itemize}
  \item $ i : 1 \to I $ parcelles
  \item $ p : 1 \to P $ plantes
  \item $ a_{ip} $ est l'abondance de la plante $p$ dans la parcelle $i$
  \item $f_{ip} = \frac{a_{ip}}{\sum_{1}^{P} a_{ip}}$ est la fréquence relative
    de la plante $p$ dans la parcelle $i$
\end{itemize}

Je cherche $ k: 1 \to K $ groupes définis par les $ f^{k}_{p} $ tels que chaque
parcelle (définie par ses fréquences relatives) soit la combinaison de ces
groupes :

\[
  f_{ip} = \sum_{k=1}^{K} \alpha _{ik}f_{p}^{k}
\]

On définit donc chaque parcelle $i$ comme un mélange des différentes populations
<<~typiques~>>.

On prend le pari que les $K$ groupes sont des groupes écologiques stables
(\textit{i.e.} les proportions ne changent pas) et que donc l'effet de la flore
d'une parcelle sur le rendement est une combinaison des effets des différents
groupes.
On s'attend aussi que cet effet soit d'autant plus grand que l'abondance totale
est forte.
On s'attend donc que l'effet de la flore de la parcelle soit lié aux $K$
variables correspondant aux <<~projections des $a_{ik}$~>> sur les groupes, soit
à :
\[
  \sum_{p=1}^{P} a_{ip} \alpha _{ik}
\]

\end{document}
