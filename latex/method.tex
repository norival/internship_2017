\documentclass[a4paper, 12pt]{article}

\usepackage[utf8]{inputenc}
\usepackage[T1]{fontenc}
\usepackage[francais]{babel}
\usepackage[top=2.5cm, bottom=2.5cm, left=2.5cm, right=2.5cm]{geometry}
\usepackage{graphicx}
\usepackage{xcolor} %Couleurs
\usepackage{icomma}
\usepackage[colorlinks=true,linkcolor=black,citecolor=black,urlcolor=black]{hyperref}
\usepackage{tikz}
\usepackage{tabularx}
\usepackage{parskip}

\begin{document}

On définit :

\begin{itemize}
  \item $ i : 1 \to I $ parcelles
  \item $ p : 1 \to P $ plantes
  \item $ a_{ip} $ est l'abondance de la plante $p$ dans la parcelle $i$
  \item $f_{ip} = \frac{a_{ip}}{\sum_{1}^{P} a_{ip}}$ est la fréquence relative
    de la plante $p$ dans la parcelle $i$
\end{itemize}

Je cherche $ k: 1 \to K $ groupes définis par les $ f^{k}_{p} $ tels que chaque
parcelle (définie par ses fréquences relatives) soit la combinaison de ces
groupes :

\[
  f_{ip} = \sum_{k=1}^{K} \alpha _{ik}f_{p}^{k}
\]

On définit donc chaque parcelle $i$ comme un mélange des différentes populations
<<~typiques~>>.

On prend le pari que les $K$ groupes sont des groupes écologiques stables
(\textit{i.e.} les proportions ne changent pas) et que donc l'effet de la flore
d'une parcelle sur le rendement est une combinaison des effets des différents
groupes.
On s'attend aussi que cet effet soit d'autant plus grand que l'abondance totale
est forte.
On s'attend donc que l'effet de la flore de la parcelle soit lié aux $K$
variables correspondant aux <<~projections des $a_{ik}$~>> sur les groupes, soit
à :
\[
  \sum_{p=1}^{P} a_{ip} \alpha _{ik}
\]

\end{document}
